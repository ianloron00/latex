\documentclass[10pt, twocolumn]{article}
\usepackage[english,brazil]{babel}
\usepackage[left=2cm, right=2cm, top = 1cm, down = 1 cm]{geometry}
\usepackage{graphicx}
\usepackage[center]{caption}
\usepackage{amsmath}
\usepackage[utf8]{inputenc}
\usepackage{rotating}
\usepackage{csvsimple}
\usepackage{multirow}
\begin{document}
\title{Welcome to Latex}
\author{Ian L. de Almeida}
\date{March 4}

% this is a comment.
\maketitle 
\textsc{Creating an Overleaf project from a Git repository}
\par\par Here are some cool basic commands to ease your life regarding Latex and Github! 
    
\begin{enumerate}
    \item Be sure you have a \textbf{git repository on your computer}.
    \begin{itemize}
        \item \textbf{git init} is used in order to create a new git repo.
        \par After that, code:
        \par \begin{verbatim}
            git add .
            git commit -m 'first commit'
            git remote add origin git@gitlab.domain.com:username/repository.git
            git push -u origin master
        \end{verbatim}      
        \item \textbf{git clone  [urlGit]} is used when there is already a repo in Github (the rest is the standard).
    \end{itemize}
    \item Create a \textbf{new project} on Overleaf.
    \item \textbf{Find the git link} of your git. \textit{Overleaf's Project Menu} may help you find the url.
    \item Add the git link for the project as a remote in your local project. 
    \begin{verbatim}
        cd folder
        git remote add overleaf <urlOverleaf>
    \end{verbatim}
    \item \textbf{Pull} latest Overleaf content and \textbf{merge} it on the master branch:
    \begin{verbatim}
        git checkout master
        git pull overleaf master --allow-unrelated-stories
    \end{verbatim}
    \item \textbf{Revert the merge} to get rid of existing files in Overleaf.
    \begin{verbatim}
        git revert --mainline 1 HEAD
    \end{verbatim}
    \item \textbf{Push} your project.
    \begin{verbatim}
        git push overleaf master
    \end{verbatim}
%\end{}
    \begin{thebibliography}{9}
        % Creates the fonts and enumerate them.
        \bibitem{overleafAndGit} Using Git and GitHub. 
        \par \texttt{https://pt.overleaf.com/learn/how-to/Using\_{}Git\_{}and\_{}GitHub}
        \bibitem{bibliographyManagement} Bibliography Management.
        \par \texttt{https://www.overleaf.com/learn/latex/bibliography\_{}management\_{}with\_{}bibtex}
        \bibitem{multicolumns} Multiple Columns.
        \par \texttt{https://www.overleaf.com/learn/latex/Multiple\_{}columns}
    \end{thebibliography}
\end{enumerate}
\end{document}